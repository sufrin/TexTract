\documentclass[11pt]{article}
\usepackage{RCS,verbatimstyle}
\SVN    $Id: propositions.tex 64 2012-06-26 18:58:22Z sufrin $
%
%
%
\usepackage[bottom]{footmisc}
\usepackage[mathletters]{ucs}
\usepackage[utf8x]{inputenc}
%
% Sanserif blackboard bold
%
\RequirePackage[sans]{dsfont}
\def\mathbb#1{\mathds{#1}}
%
%
%
\usepackage{japecode}
\usepackage[colorlinks,linkcolor=blue,filecolor=blue]{hyperref} 
%
%
%
\title{Propositional Logic and Proof \\\textit{via}\\ Functional Programming}
\author{Bernard Sufrin\thanks{Oxford University Computer Science Department}}
\date{Hilary Term, 2013}
%
%
%
%\codestyle{haskell}
\parindent=0pt\parskip=\medskipamount
%
%
%
\begin{document}
\maketitle
\begin{abstract}
\noindent In these notes we give an account of the
development of a simple \textit{proof assistant} for
propositional logic. Our intention is to demonstrate a useful way of
thinking about propositions and proofs as data structures.
\end{abstract}

\tableofcontents

\newpage
\section{Introduction}

\section{Propositions}

\subsection{Representation}
We represent propositions as recursive data type. The
comment by each constructor is an example of the notation
used to write that kind of proposition.

\begin{code}[propositions.hs]
module Propositions where
  
  ... Import types classes and functions from the Prelude

  data Prop = Atom Atomic         -- a, b ...
            | Not  Prop           -- not p
            | Prop `And` Prop     -- p ∧ q
            | Prop `Or`  Prop     -- p ∨ q
            | Prop `Imp` Prop     -- p ⇒ q
            | Prop `Iff` Prop     -- p ⇔ q
            deriving (Eq)
\end{code}

\subsection{Haskell notation}
We want to make it convenient to write propositions in Haskell itself; so we 
define Haskell operator symbols corresponding to appropriate
constructors of the \textsf{Prop} type, and give them conventional
syntactic properties.\footnote{It is unfortunate that
we can do this only for infix operators, for we would
have liked to use $\lnot$ as a negation symbol; but there is
no way of defining a \textit{symbolic} prefix operator in Haskell.} 

\begin{code}[]
  infixl 9 ∧;  (∧)  = And
  infixl 8 ∨;  (∨)  = Or
  infixr 7 ⇒; (⇒)  = Imp
  infixl 6 ⇔; (⇔)  = Iff  
  not = Not
\end{code}

We are not interested in the internal structure of atomic
propositions, but it will be convenient to represent
them as single letters. Here we invent Haskell names for a few atomic
propositions.
\begin{code}
  type Atomic = String
  [a,b,c,d,p,q,r,s] = map (Atom . (:"")) "abcdpqrs"
\end{code}


\subsection{Printing}
We want propositions to be printed using the same notation
that we write them in (in Haskell). We declare \textsf{Prop}
to be an instance of the \textbf{Show} class, and define a
function \textsf{showProp} that implements
\textsf{showsPrec} for propositions.

\begin{code}[]
  instance (Show Prop) where showsPrec = showProp  
  showProp::  Int -> Prop -> ShowS
  showProp d (Atom a)    = showString a
  showProp d (Not  p)    = showString "not " . showProp 10 p
  showProp d (p `And` q) = showInfix d 9 9 " ∧ "     p q
  showProp d (p `Or` q)  = showInfix d 8 8 " ∨ "     p q
  showProp d (p `Imp` q) = showInfix d 7 6 " ⇒ "    p q
  showProp d (p `Iff` q) = showInfix d 5 5 " ⇔ "    p q
\end{code}

The auxiliary function \textsf{showInfix} handles the
showing of binary composite propositions, and takes account
of the right-associativity of implication in the usual way,
namely by giving implication a slightly higher syntactic
priority on its right than on its left. 

\begin{code}
  showInfix:: Int -> Int -> Int -> String -> Prop -> Prop -> ShowS
  showInfix d d' d'' op l r  =  
            showParen (d>=d') $ 
            showProp  d' l   .
            showString op    . 
            showProp  d'' r
\end{code}

Here are some examples of propositions

\begin{code}[]
  p0 = (a ⇒ b) ⇒ c
  p1 = a ⇒ b ⇒ c
  p2 = a ∧ b ⇒ c ∧ d
  p3 = not a  ∧  b  ⇒  not c  ∧  d
  p4 = (a ⇒ (b ⇒ c)) ⇒ ((a ⇒ b) ⇒ c)
  p5 = not p4
\end{code}

and here's what the Haskell interpreter prints for them. Notice that the redundant
parentheses on the right of $p4$ are omitted, but that in all other respects 
the Haskell output was identical to the Haskell input.
\begin{verbatim}
  > [p0, p1, p2, p3, p4, p5]
  [       (a ⇒ b) ⇒ c
  ,       a ⇒ b ⇒ c
  ,       a ∧ b ⇒ c ∧ d
  ,       not a ∧ b ⇒ not c ∧ d
  ,       (a ⇒ b ⇒ c) ⇒ (a ⇒ b) ⇒ c
  ,       not ((a ⇒ b ⇒ c) ⇒ (a ⇒ b) ⇒ c)
  ]
\end{verbatim}


\section{Proof Representation}
\begin{code}
  newtype Sequent = Sequent ([Prop], Prop)
  infix 1 |-
  hyps |- conc = Sequent (hyps, conc)
      
  instance (Show Sequent) where 
     show (Sequent(hyps, conc)) = showHyps hyps ++ " |- " ++ show conc
     
  data    Proof   = Proven [Proof] Sequent Rule
                  | Gap
                  
  instance (Show Proof) where
     showsPrec = showProof 
     
  showHyps []     = ""
  showHyps [h]    = show h
  showHyps (h:hs) = show h ++  ", " ++ showHyps hs
  
  showProof :: Int -> Proof -> ShowS
  showProof level Gap = indentBy level . showString "..."
  showProof level (Proven subproofs seq rule) =
            foldAll showSubproof subproofs .
            indentBy level                 .
            showsPrec 0 seq                .
            showString " by "              .
            showString rule
            where 
            showSubproof proof s = showProof (level+1) proof ("\n" ++ s)
            foldAll f xs ys = foldr f ys xs         
                  
  indentBy:: Int -> ShowS
  indentBy level s = take (2*level) spaces ++ s
  
  spaces:: String
  spaces = ' ':spaces 
  
  type Rule = String

  a1 = Proven [] [p, q] |- p
  
  goal1  = [p ⇒ (q ⇒ r)] |- p ⇒ r
  proof1 = Proven [proof2] goal1 "|-=>"

  
\end{code}

\section{Proof Rules}

\section{Related Work}

\href{http://pauillac.inria.fr/~leifer/articles/logic/LogicPearl.dvi.gz}
{Deduction for functional programmers} by James J. Leifer and Bernard Sufrin.

Journal of Functional Programming, volume 6, number 2, 1996.
\begin{code}[... Prelude]

  {- This is just some nonsense to test
     multiple additions to the same chunk
  -}

\end{code}
This paper was the result of research we did together while James Leifer was an undergraduate 
in Oxford. It describes a new way of
teaching formal deductive logic to students familiar with functional
programming and introduces to them metamathematical ideas by
exploiting the students' familiarity with freely-generated
data-structures and their recursion and induction principles.

\href{http://pauillac.inria.fr/~leifer/articles/logic/LogicviaFP.300.ps.gz}
{Formal logic via functional programming} by James J. Leifer (June 1995).
Final year dissertation, Oxford University, supervised by
C.A.R. Hoare and Bernard Sufrin.

This report is a greatly expanded version of the above paper. See
the
\href{http://pauillac.inria.fr/~leifer/articles/logic/LogicviaFP.Errata}
{Errata} list, the Gofer\footnote{Haskell's immediate predecessor}
\href{http://pauillac.inria.fr/~leifer/articles/logic/LogicviaFP.gs}
{source-listings} extracted from the document, and the
\href{http://pauillac.inria.fr/~leifer/articles/logic/LogicviaFP.j} {JAPE
source-listings} (which are no longer compatible with the
latest version of JAPE).

\appendix
\section{Deferred sections of code}

\subsection{... import types classes and functions from the Prelude}
\begin{code}[... import types classes and functions from the Prelude]
  ... Prelude
  import Prelude(Eq, String, Show, ShowS, Int, 
                 map, (.), (++), foldr, take, unlines, (>=), (+), (*), ($), 
                 showParen, showString, showsPrec, show)

\end{code}
\end{document}





